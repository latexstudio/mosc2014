\documentclass{beamer}
%\usetheme{lankton-keynote}
%\usetheme{Gombak} ni cerah
\usetheme{Ampang}% ni gelap

\usepackage{ubuntu,caption}
\usepackage{beamerthemesplitku}

%\usepackage{beamerthemetree}
%\usecolortheme{beaver}
\usecolortheme{ampangcolor}
%\usecolortheme{warna}
\usepackage{graphicx}
\usepackage{tikz}
\usepackage{forest}
\usetikzlibrary{arrows,shapes,positioning,shadows,trees}

%\newtheorem{algorithm}{Algorithm}[section] 
\usepackage{float}
\floatstyle{ruled}


\usepackage{fancyvrb}
\usepackage{tabularx}
\usepackage{graphicx}
\usepackage{longtable}
%\usepackage{ccaption} 
\usepackage{listings}

\usepackage{color}

%\setbeamercolor{frametitle}{black!100,fg=white}
%\setbeamercolor{titlelike}{parent=structure,bg=red!60!blue!90!green!95,fg=white}
%\setbeamercolor*{palette primary}{use=structure,bg=red!60!blue!90!green!95,fg=white}
%\setbeamercolor*{palette quaternary}{use=structure,bg=red!10!black!100,fg=white}


\makeatletter
\def\PY@reset{\let\PY@it=\relax \let\PY@bf=\relax%
    \let\PY@ul=\relax \let\PY@tc=\relax%
    \let\PY@bc=\relax \let\PY@ff=\relax}
\def\PY@tok#1{\csname PY@tok@#1\endcsname}
\def\PY@toks#1+{\ifx\relax#1\empty\else%
    \PY@tok{#1}\expandafter\PY@toks\fi}
\def\PY@do#1{\PY@bc{\PY@tc{\PY@ul{%
    \PY@it{\PY@bf{\PY@ff{#1}}}}}}}
\def\PY#1#2{\PY@reset\PY@toks#1+\relax+\PY@do{#2}}

\expandafter\def\csname PY@tok@gd\endcsname{\def\PY@tc##1{\textcolor[rgb]{0.63,0.00,0.00}{##1}}}
\expandafter\def\csname PY@tok@gu\endcsname{\let\PY@bf=\textbf\def\PY@tc##1{\textcolor[rgb]{0.50,0.00,0.50}{##1}}}
\expandafter\def\csname PY@tok@gt\endcsname{\def\PY@tc##1{\textcolor[rgb]{0.00,0.27,0.87}{##1}}}
\expandafter\def\csname PY@tok@gs\endcsname{\let\PY@bf=\textbf}
\expandafter\def\csname PY@tok@gr\endcsname{\def\PY@tc##1{\textcolor[rgb]{1.00,0.00,0.00}{##1}}}
\expandafter\def\csname PY@tok@cm\endcsname{\let\PY@it=\textit\def\PY@tc##1{\textcolor[rgb]{0.25,0.50,0.50}{##1}}}
\expandafter\def\csname PY@tok@vg\endcsname{\def\PY@tc##1{\textcolor[rgb]{0.10,0.09,0.49}{##1}}}
\expandafter\def\csname PY@tok@m\endcsname{\def\PY@tc##1{\textcolor[rgb]{0.40,0.40,0.40}{##1}}}
\expandafter\def\csname PY@tok@mh\endcsname{\def\PY@tc##1{\textcolor[rgb]{0.40,0.40,0.40}{##1}}}
\expandafter\def\csname PY@tok@go\endcsname{\def\PY@tc##1{\textcolor[rgb]{0.53,0.53,0.53}{##1}}}
\expandafter\def\csname PY@tok@ge\endcsname{\let\PY@it=\textit}
\expandafter\def\csname PY@tok@vc\endcsname{\def\PY@tc##1{\textcolor[rgb]{0.10,0.09,0.49}{##1}}}
\expandafter\def\csname PY@tok@il\endcsname{\def\PY@tc##1{\textcolor[rgb]{0.40,0.40,0.40}{##1}}}
\expandafter\def\csname PY@tok@cs\endcsname{\let\PY@it=\textit\def\PY@tc##1{\textcolor[rgb]{0.25,0.50,0.50}{##1}}}
\expandafter\def\csname PY@tok@cp\endcsname{\def\PY@tc##1{\textcolor[rgb]{0.74,0.48,0.00}{##1}}}
\expandafter\def\csname PY@tok@gi\endcsname{\def\PY@tc##1{\textcolor[rgb]{0.00,0.63,0.00}{##1}}}
\expandafter\def\csname PY@tok@gh\endcsname{\let\PY@bf=\textbf\def\PY@tc##1{\textcolor[rgb]{0.00,0.00,0.50}{##1}}}
\expandafter\def\csname PY@tok@ni\endcsname{\let\PY@bf=\textbf\def\PY@tc##1{\textcolor[rgb]{0.60,0.60,0.60}{##1}}}
\expandafter\def\csname PY@tok@nl\endcsname{\def\PY@tc##1{\textcolor[rgb]{0.63,0.63,0.00}{##1}}}
\expandafter\def\csname PY@tok@nn\endcsname{\let\PY@bf=\textbf\def\PY@tc##1{\textcolor[rgb]{0.00,0.00,1.00}{##1}}}
\expandafter\def\csname PY@tok@no\endcsname{\def\PY@tc##1{\textcolor[rgb]{0.53,0.00,0.00}{##1}}}
\expandafter\def\csname PY@tok@na\endcsname{\def\PY@tc##1{\textcolor[rgb]{0.49,0.56,0.16}{##1}}}
\expandafter\def\csname PY@tok@nb\endcsname{\def\PY@tc##1{\textcolor[rgb]{0.00,0.50,0.00}{##1}}}
\expandafter\def\csname PY@tok@nc\endcsname{\let\PY@bf=\textbf\def\PY@tc##1{\textcolor[rgb]{0.00,0.00,1.00}{##1}}}
\expandafter\def\csname PY@tok@nd\endcsname{\def\PY@tc##1{\textcolor[rgb]{0.67,0.13,1.00}{##1}}}
\expandafter\def\csname PY@tok@ne\endcsname{\let\PY@bf=\textbf\def\PY@tc##1{\textcolor[rgb]{0.82,0.25,0.23}{##1}}}
\expandafter\def\csname PY@tok@nf\endcsname{\def\PY@tc##1{\textcolor[rgb]{0.00,0.00,1.00}{##1}}}
\expandafter\def\csname PY@tok@si\endcsname{\let\PY@bf=\textbf\def\PY@tc##1{\textcolor[rgb]{0.73,0.40,0.53}{##1}}}
\expandafter\def\csname PY@tok@s2\endcsname{\def\PY@tc##1{\textcolor[rgb]{0.73,0.13,0.13}{##1}}}
\expandafter\def\csname PY@tok@vi\endcsname{\def\PY@tc##1{\textcolor[rgb]{0.10,0.09,0.49}{##1}}}
\expandafter\def\csname PY@tok@nt\endcsname{\let\PY@bf=\textbf\def\PY@tc##1{\textcolor[rgb]{0.00,0.50,0.00}{##1}}}
\expandafter\def\csname PY@tok@nv\endcsname{\def\PY@tc##1{\textcolor[rgb]{0.10,0.09,0.49}{##1}}}
\expandafter\def\csname PY@tok@s1\endcsname{\def\PY@tc##1{\textcolor[rgb]{0.73,0.13,0.13}{##1}}}
\expandafter\def\csname PY@tok@sh\endcsname{\def\PY@tc##1{\textcolor[rgb]{0.73,0.13,0.13}{##1}}}
\expandafter\def\csname PY@tok@sc\endcsname{\def\PY@tc##1{\textcolor[rgb]{0.73,0.13,0.13}{##1}}}
\expandafter\def\csname PY@tok@sx\endcsname{\def\PY@tc##1{\textcolor[rgb]{0.00,0.50,0.00}{##1}}}
\expandafter\def\csname PY@tok@bp\endcsname{\def\PY@tc##1{\textcolor[rgb]{0.00,0.50,0.00}{##1}}}
\expandafter\def\csname PY@tok@c1\endcsname{\let\PY@it=\textit\def\PY@tc##1{\textcolor[rgb]{0.25,0.50,0.50}{##1}}}
\expandafter\def\csname PY@tok@kc\endcsname{\let\PY@bf=\textbf\def\PY@tc##1{\textcolor[rgb]{0.00,0.50,0.00}{##1}}}
\expandafter\def\csname PY@tok@c\endcsname{\let\PY@it=\textit\def\PY@tc##1{\textcolor[rgb]{0.25,0.50,0.50}{##1}}}
\expandafter\def\csname PY@tok@mf\endcsname{\def\PY@tc##1{\textcolor[rgb]{0.40,0.40,0.40}{##1}}}
\expandafter\def\csname PY@tok@err\endcsname{\def\PY@bc##1{\setlength{\fboxsep}{0pt}\fcolorbox[rgb]{1.00,0.00,0.00}{1,1,1}{\strut ##1}}}
\expandafter\def\csname PY@tok@kd\endcsname{\let\PY@bf=\textbf\def\PY@tc##1{\textcolor[rgb]{0.00,0.50,0.00}{##1}}}
\expandafter\def\csname PY@tok@ss\endcsname{\def\PY@tc##1{\textcolor[rgb]{0.10,0.09,0.49}{##1}}}
\expandafter\def\csname PY@tok@sr\endcsname{\def\PY@tc##1{\textcolor[rgb]{0.73,0.40,0.53}{##1}}}
\expandafter\def\csname PY@tok@mo\endcsname{\def\PY@tc##1{\textcolor[rgb]{0.40,0.40,0.40}{##1}}}
\expandafter\def\csname PY@tok@kn\endcsname{\let\PY@bf=\textbf\def\PY@tc##1{\textcolor[rgb]{0.00,0.50,0.00}{##1}}}
\expandafter\def\csname PY@tok@mi\endcsname{\def\PY@tc##1{\textcolor[rgb]{0.40,0.40,0.40}{##1}}}
\expandafter\def\csname PY@tok@gp\endcsname{\let\PY@bf=\textbf\def\PY@tc##1{\textcolor[rgb]{0.00,0.00,0.50}{##1}}}
\expandafter\def\csname PY@tok@o\endcsname{\def\PY@tc##1{\textcolor[rgb]{0.40,0.40,0.40}{##1}}}
\expandafter\def\csname PY@tok@kr\endcsname{\let\PY@bf=\textbf\def\PY@tc##1{\textcolor[rgb]{0.00,0.50,0.00}{##1}}}
\expandafter\def\csname PY@tok@s\endcsname{\def\PY@tc##1{\textcolor[rgb]{0.73,0.13,0.13}{##1}}}
\expandafter\def\csname PY@tok@kp\endcsname{\def\PY@tc##1{\textcolor[rgb]{0.00,0.50,0.00}{##1}}}
\expandafter\def\csname PY@tok@w\endcsname{\def\PY@tc##1{\textcolor[rgb]{0.73,0.73,0.73}{##1}}}
\expandafter\def\csname PY@tok@kt\endcsname{\def\PY@tc##1{\textcolor[rgb]{0.69,0.00,0.25}{##1}}}
\expandafter\def\csname PY@tok@ow\endcsname{\let\PY@bf=\textbf\def\PY@tc##1{\textcolor[rgb]{0.67,0.13,1.00}{##1}}}
\expandafter\def\csname PY@tok@sb\endcsname{\def\PY@tc##1{\textcolor[rgb]{0.73,0.13,0.13}{##1}}}
\expandafter\def\csname PY@tok@k\endcsname{\let\PY@bf=\textbf\def\PY@tc##1{\textcolor[rgb]{0.00,0.50,0.00}{##1}}}
\expandafter\def\csname PY@tok@se\endcsname{\let\PY@bf=\textbf\def\PY@tc##1{\textcolor[rgb]{0.73,0.40,0.13}{##1}}}
\expandafter\def\csname PY@tok@sd\endcsname{\let\PY@it=\textit\def\PY@tc##1{\textcolor[rgb]{0.73,0.13,0.13}{##1}}}

\def\PYZbs{\char`\\}
\def\PYZus{\char`\_}
\def\PYZob{\char`\{}
\def\PYZcb{\char`\}}
\def\PYZca{\char`\^}
\def\PYZam{\char`\&}
\def\PYZlt{\char`\<}
\def\PYZgt{\char`\>}
\def\PYZsh{\char`\#}
\def\PYZpc{\char`\%}
\def\PYZdl{\char`\$}
\def\PYZhy{\char`\-}
\def\PYZsq{\char`\'}
\def\PYZdq{\char`\"}
\def\PYZti{\char`\~}
% for compatibility with earlier versions
\def\PYZat{@}
\def\PYZlb{[}
\def\PYZrb{]}
\makeatother




\setbeamertemplate{navigation symbols}{} %no nav symbols
%\logo{\includegraphics[width=12.6cm,height=1cm]{kul.png}}
\logo{\includegraphics[scale=0.2]{mosc.jpg}}
%\logo{\includegraphics[scale=0.15]{cambridge.png}}
\title[MOSC 2014\hspace{2em}\insertframenumber/\inserttotalframenumber]{\fontUbuntuCondensed Android Custom Kernel/ROM design}
\author[Muhammad Najmi]{Muhammad Najmi Ahmad Zabidi\\
[1.7ex]\tiny{\textit{IIUM}}}
\institute{MOSC 2014\\Menara SSM \\ Kuala Lumpur, Malaysia}
%\institute{FSKSM, UTM Skudai, JB}

\date{24-25 September 2014}

\begin{document} 
\setbeamerfont{frametitle}{size=\small}
\maketitle

\lstset{basicstyle=\ttfamily\tiny}

\begin{frame}{About}
 \begin{itemize}
  \item I am a research grad student in Universiti Teknologi Malaysia, Skudai, Johor Bahru, Malaysia
  \item My current employer is International Islamic University Malaysia, Kuala Lumpur
  \item Research area - malware detection, narrowing on Windows executables
  \item Doing things on Android kernel and ROM due to some stories...
 \end{itemize}

\end{frame}

\begin{frame}{A bit about Android}

\begin{columns}[t]
 \begin{column}{5cm}
  
 \includegraphics[scale=0.35]{linux-android.jpg}
 
 
 \end{column}


\begin{column}{5cm}
 
 
 %\section{Intro}
 \begin{itemize}
  \item Android is a mobile operating system
  \item Using \emph{Linux} kernel
  \item Components for kernel are C language
  \item Components for interface are mostly C++ and Java
 \end{itemize}

 
\end{column}

 \end{columns}
 
\end{frame}




\begin{frame}{Comparison between Android kernel and ROM}

\begin{table}
 \begin{tabular} { |p{5cm}| p{5cm}| }
 \hline
  Kernel&ROM\\
  \hline
  
  GPL licensed & Apache licensed\\
  \hline
  
  Source code must be published & Source code is not compulsory to be published.
  Hence any modifications are not neccessarily going back to the public\\
  
  \hline
  
 \end{tabular}

\end{table}
\end{frame}


\begin{frame}{Android structure}
 \includegraphics[scale=0.3]{android.jpg}
\end{frame}

\begin{frame}
%\frametitle{A first slide}

\begin{center}
\Huge Custom Kernel
\end{center}
\end{frame}


\begin{frame}{Android kernel vs Linux kernel}
 \includegraphics[scale=0.6]{kernel.png}
 
 \tiny Source: \url{http://eecatalog.com/embeddedlinux/2011/08/23/from-zero-to-boot-porting-android-to-your-arm-platform/}
\end{frame}


\begin{frame}{My custom Android kernels}
 \begin{itemize}
  \item Some are based from AOSP (Android Open Source Project) kernels - original source are from Google's git
  \item Some are based from CM (AOSP + Code Aurora Forum (CAF) commits)
  \item Some are based from other custom kernels which are based from two sources above
 \end{itemize}

\end{frame}

\begin{frame}
 \begin{itemize}
  \item I developed my custom kernels for two devices
  \begin{itemize}
   \item Nexus 4 (codename: Mako)
   \item \pause Nexus 5 (codename: Hammerhead) 
   \begin{itemize}
    \item \pause After I sold my Mako :)
   \end{itemize}

  \end{itemize}

 \end{itemize}

\end{frame}


\begin{frame}{Why use custom kernel}
 
 Customization,add-on features:

 \begin{itemize}
  \item \pause Sound patch (for e.g: Faux sound patch)
  \item \pause Allow DoubleTaptoWake(DT2W) or Sweep2Wake, Sweep2Sleep (S2W,S2S) features
  \item \pause Allow many more CPU governors to be used
  \item \pause Allow under/overvolting
  \item \pause Allow number of online/offline CPUs using many methods
  \item \pause Allow many more TCP congestion methods
 \end{itemize}
  
\end{frame}

\begin{frame}{Skillsets for kernel modifying/developing}
\begin{itemize}
 \item Git knowledge
 \begin{itemize}
  \item Knows at least how to clone, pull, push
  \item Then reading git log.. (i'm using --pretty option)
  \item Creating branch, reset to certain checkpoint/offset.. resetting everything (git reset --hard)
 \end{itemize}

 
\end{itemize}

 
\end{frame}

%\section{Git}
\begin{frame}[fragile]{Git cloning the source}
 \tiny{\begin{verbatim}
najmi@quds:~$ git clone https://android.googlesource.com/kernel/msm -b android-msm-hammerhead-3.4-l-preview
Cloning into 'msm'...
remote: Sending approximately 953.94 MiB ...
remote: Finding sources: 100% (3604873/3604873)
Receiving objects:   0% (14589/3604873), 4.63 MiB | 656.00 KiB/s
 \end{verbatim}}
\end{frame}

\begin{frame}[fragile]
\tiny
\begin{Verbatim}[commandchars=\\\{\}]
najmi@quds:\PYZti{}/cempaka\PYZhy{}kernel\PY{n+nv}{\PYZdl{} }git log \PYZhy{}\PYZhy{}pretty\PY{o}{=}format:
\PY{l+s+s1}{\PYZsq{}\PYZpc{}Cred\PYZpc{}h\PYZpc{}Creset \PYZhy{}\PYZpc{}C(yellow)\PYZpc{}d\PYZpc{}Creset \PYZpc{}s \PYZpc{}Cgreen(\PYZpc{}cr)
\PYZpc{}C(bold blue)\PYZlt{}\PYZpc{}an\PYZgt{}\PYZpc{}Creset
\PYZsq{}} \PYZhy{}\PYZhy{}abbrev\PYZhy{}commit
ba22633 \PYZhy{} \PY{o}{(}HEAD, origin/cempaka\PYZhy{}stable, cempaka\PYZhy{}stable\PY{o}{)}
Cempaka v2.5 \PY{o}{(}11 days ago\PY{o}{)} \PYZlt{}Muhammad Najmi Ahmad Zabidi\PYZgt{}
acaaeea \PYZhy{} Merge branch \PY{l+s+s1}{\PYZsq{}ElementalX\PYZhy{}1.00\PYZhy{}cm\PYZsq{}} of
https://github.com/flar2/ElementalX\PYZhy{}N5 into cempaka\PYZhy{}stable \PY{o}{(}11 days ago\PY{o}{)} \PYZlt{}Muhammad Najmi Ahmad Zabidi\PYZgt{}
039a263 \PYZhy{} \PY{o}{(}elementalx/ElementalX\PYZhy{}1.00\PYZhy{}cm\PY{o}{)} Merge branch \PY{l+s+s1}{\PYZsq{}ElementalX\PYZhy{}1.00
\PYZsq{}} into ElementalX\PYZhy{}1.00\PYZhy{}cm 
\PY{o}{(}12 days ago\PY{o}{)} \PYZlt{}flar2\PYZgt{}
cb7e4f0 \PYZhy{} \PY{o}{(}elementalx/ElementalX\PYZhy{}1.00\PY{o}{)} update defconfig \PY{o}{(}12 days ago\PY{o}{)} \PYZlt{}flar2\PYZgt{}
753de48 \PYZhy{} msm\PYZhy{}sleeper: use ex\PYZus{}max\PYZus{}freq \PY{o}{(}12 days ago\PY{o}{)} \PYZlt{}flar2\PYZgt{}
c5cc9d2 \PYZhy{} Merge branch \PY{l+s+s1}{\PYZsq{}ElementalX\PYZhy{}1.00\PYZsq{}} into
ElementalX\PYZhy{}1.00\PYZhy{}cm \PY{o}{(}2 weeks ago\PY{o}{)} \PYZlt{}flar2\PYZgt{}
8438630 \PYZhy{} vibrator: change permissions again \PY{o}{(}2 weeks ago\PY{o}{)} \PYZlt{}flar2\PYZgt{}
e51fa2d \PYZhy{} Revert \PY{l+s+s2}{\PYZdq{}vibrator: change sysfs permissions\PYZdq{}} \PY{o}{(}2 weeks ago\PY{o}{)} \PYZlt{}flar2\PYZgt{}
9949aff \PYZhy{} Merge branch \PY{l+s+s1}{\PYZsq{}ElementalX\PYZhy{}1.00\PYZsq{}} into ElementalX\PYZhy{}1.00\PYZhy{}cm \PY{o}{(}3 weeks ago\PY{o}{)} \PYZlt{}flar2\PYZgt{}
cbee9fe \PYZhy{} update defconfig \PY{o}{(}3 weeks ago\PY{o}{)} \PYZlt{}flar2\PYZgt{}
\end{Verbatim}
\end{frame}

\begin{frame}
 \begin{itemize}
  \item Most of the developers' works are hosted on github
  \item Some use sourceforge's git and bitbucket's 
  \item I prefer github because I am familiar with it
 \end{itemize}

\end{frame}


%\begin{frame}
%\tikzstyle{decision} = [diamond, draw, fill=blue!20, 
%    text width=4.5em, text badly centered, node distance=3cm, inner sep=0pt]
%\tikzstyle{block} = [rectangle, draw, fill=red!80!green!80!blue!80, 
%    text width=5em, text centered, rounded corners, minimum height=4em]
%\tikzstyle{line} = [draw, -latex']
%\tikzstyle{cloud} = [draw, ellipse,fill=red!20, node distance=3cm,
%    minimum height=2em]
%    \tiny
%\begin{tikzpicture}[node distance = 2cm, auto]
    % Place nodes
  
  %\node [block] (AOSP) {AOSP kernel};
  % \node [cloud, left of=init] (expert) {expert};
  % \node [cloud, right of=init] (system) {system};
   % \node [block,  fill=red!50!green!80!blue!100,  below right 2cm of=AOSP, node distance=4cm] (franco) {Franco gombak kernel};
   % \node [block,  fill=red!50!green!80!blue!100,  below of=AOSP, node distance=2cm] (ampang) {Ampang kernel};
   % \node [block,  fill=red!50!green!80!blue!100,  below right of=AOSP, node distance=2cm] (hellscore) {Hellscore kernel};
   % \node [block,  fill=red!50!green!80!blue!100,  below left of=AOSP, node distance=2cm] (semaphore) {Semaphore kernel};
   % \node [block, fill=red!80!green!80!blue!100,  below of =hellscore, node distance=2cm] (zulfa) {Zulfa kernel};
   % \node [block, fill=red!80!green!80!blue!100,  below of=semaphore, node distance=2cm] (pandan) {Pandan kernel};
   %\node [block, below of=identify] (evaluate) {evaluate candidate models};
    %node [block, left of=evaluate, node distance=3cm] (update) {update model};
   %\node [decision, below of=evaluate] (decide) {is best candidate better?};
   %\node [block, below of=decide, node distance=3cm] (stop) {stop};
    % Draw edges
   % \path [line] (AOSP) -- (ampang);
   % \path [line] (AOSP) -- (hellscore);
   % \path [line] (AOSP) -- (semaphore);
   % \path [line] (AOSP) -- (franco);
   % \path [line] (hellscore) -- (zulfa);
   % \path [line] (semaphore) -- (pandan);
    %path [line] (identify) -- (evaluate);
    %path [line] (evaluate) -- (decide);
    %path [line] (decide) -| node [near start] {yes} (update);
    %path [line] (update) |- (identify);
    %path [line] (decide) -- node {no}(stop);
    %path [line,dashed] (expert) -- (init);
    %path [line,dashed] (system) -- (init);
    %path [line,dashed] (system) |- (evaluate);
%\end{tikzpicture}
%\end{frame}

\begin{frame}{Android timeline, from 4.x}
\begin{itemize}
 \item 4.0 (Ice Cream Sandwich)
  \item 4.1 (Jelly Bean)
  \item 4.2  (Jelly Bean)
  \item 4.3  (Jelly Bean)
  \item 4.4 (KitKat)
  \begin{itemize}
   \item 4.4.1
   \item 4.4.2
   \item 4.4.3
   \item 4.4.4 (latest, as of now)
  \end{itemize}

\end{itemize}

 
\end{frame}

%\section{My works}
%\subsection{Nexus 4/Mako}
\begin{frame}{My kernel projects for Nexus 4/Mako}
 \tikzset{
  basic/.style  = {draw, text width=3cm, drop shadow, rectangle},%font=\sffamily, rectangle},
  root/.style   = {basic, rounded corners=1pt, thin, align=center,
                   fill=red!40!blue!80!green!100},
  level 2/.style = {basic, rounded corners=1pt, thin,align=center, fill=red!30!blue!50!green!100,
                   text width=8em},
  level 3/.style = {basic, thin, align=left, fill=red!70!blue!90!green!100,text width=8em}
}

\begin{tikzpicture}[
  level 1/.style={sibling distance=20mm},
  edge from parent/.style={->,draw},
  >=latex]
\tiny
% root of the the initial tree, level 1
\node[root] (root) {Kernel Sources}
% The first level, as children of the initial tree
  child {node[level 2,text width=6em] (c1) {Semaphore Kernel}}
  child {node[level 2,text width=4.5em] (c2) {Hellscore Kernel}}  
  child {node[level 2,text width=4em] (c4) {Franco Kernel}}
  child {node[level 2, fill=red!90!blue!90!green!90,text width=2em] (c5) {CM}} 
  child {node[level 2,text width=4em] (c6) {Bricked Kernel}}
  child {node[level 2,text width=4em] (c3) {Ampang Kernel}}  
  ;
  
  
% The second level, relatively positioned nodes
\begin{scope}[every node/.style={level 3}]
\node [below of = c1, xshift=15pt,text width=4em] (c11) {Pandan Kernel};
\node [below of = c2, xshift=15pt,text width=4em] (c21) {Zulfa Kernel};
\node [below of = c4, xshift=15pt,text width=4em] (c41) {Franco Gombak Kernel};
\node [below of = c5, xshift=15pt,text width=4em] (c51) {Aufa Kernel};
\node [below of = c6, xshift=15pt,text width=4em] (c61) {Seladang Kernel};

\end{scope}

% lines from each level 1 node to every one of its "children"
%\foreach \value in {1,2,3}
\foreach \value in {1}
  \draw[->] (c1.220) |- (c1\value.west);

%\foreach \value in {1,...,4}
\foreach \value in {1}
  \draw[->] (c2.220) |- (c2\value.west);

\foreach \value in {1}
  \draw[->] (c4.220) |- (c4\value.west);

 % \foreach \value in {1}
 % \draw[->] (c5.240) |- (c5\value.east);  
  
\foreach \value in {1}
  \draw[->] (c6.220) |- (c6\value.west);
  
\foreach \value in {1}
  \draw[->] (c5.220) |- (c5\value.west);  

  
\end{tikzpicture}
\end{frame}



%\subsection{Nexus 5/Hammerhead}
\begin{frame}{My kernel projects for Nexus 5/Hammerhead}
 \tikzset{
  basic/.style  = {draw, text width=3cm, drop shadow, rectangle},%font=\sffamily, rectangle},
  root/.style   = {basic, rounded corners=1pt, thin, align=center,
                   fill=red!40!blue!80!green!100},
  level 2/.style = {basic, rounded corners=1pt, thin,align=center, fill=red!30!blue!50!green!100,
                   text width=10em},
  level 3/.style = {basic, thin, align=left, fill=red!70!blue!90!green!100,text width=10em}
}

\begin{tikzpicture}[
  level 1/.style={sibling distance=30mm},
  edge from parent/.style={->,draw},
  >=latex]
\tiny
% root of the the initial tree, level 1
\node[root] {Kernel Sources}
% The first level, as children of the initial tree
  child {node[level 2] (c1) {ElementalX}}
  child {node[level 2] (c2) {CodeBlue}}
  child {node[level 2] (c3) {Jerung Kernel}}
  child {node[level 2, fill=red!90!blue!90!green!90] (c4) {Lekiu Kernel (for L preview)}} 
  ; 
% The second level, relatively positioned nodes
\begin{scope}[every node/.style={level 3}]
\node [below of = c1, xshift=15pt] (c11) {Cempaka Kernel};
\node [below of = c2, xshift=15pt] (c21) {Blue Kelisa Kernel};

\end{scope}

% lines from each level 1 node to every one of its "children"
%\foreach \value in {1,2,3}
\foreach \value in {1}
  \draw[->] (c1.190) |- (c1\value.west);

%\foreach \value in {1,...,4}
\foreach \value in {1}
  \draw[->] (c2.190) |- (c2\value.west);

%\foreach \value in {1}
%  \draw[->] (c4.195) |- (c4\value.west);
  

  
\end{tikzpicture}
\end{frame}
 
 \begin{frame}{My Nexus 4/Mako Kernel Features}
  \tiny
 \begin{table}

 \begin{tabular} {| p{1cm}|p{1cm}|p{1cm}|p{1cm}|p{0.7cm}|p{1.25cm}|p{1cm}|p{1cm}|}
   
 \hline
  Kernel&Tap2Wake and S2S/S2W&Additional Governors&Additional Schedulers&Fast Charge&Multiboot (kexec)&Intelliplug&Advanced MPD\\
  \hline
  Aufa&X&X&X&X&X&X&- \\
  \hline
  Zulfa&X&X&X&X&X&X&-\\
  \hline
  Ampang&X&X&X&X&X&X&-\\
  \hline
  Franco Gombak&X&X&X&X&X&-&-\\
  \hline
  Seladang&X&X&X&X&X&-&X\\
  \hline
  Pandan&X&X&X&X&X&X&-\\
  \hline
  
 \end{tabular}
\caption{Features of the custom kernels (at least from what I remember)}
 \end{table}  
 \end{frame}

 \begin{frame}{My Nexus 5/Hammerhead Kernel Features}
  \tiny
 \begin{table}

 \begin{tabular} {| p{1.3cm}|p{.8cm}|p{1cm}|p{1.2cm}|p{0.7cm}|p{1.25cm}|p{1cm}|p{.8cm}|}
   
 \hline
  Kernel&Tap2Wake and S2S/S2W&Additional Governors&Additional Schedulers&Fast Charge&Multiboot (kexec)&Intelliplug&Advanced MPD\\
  \hline
  Cempaka&X&X&X&X&X&X&- \\
  \hline
  Blue Kelisa&X&X&X&X&X&X&-\\
  \hline
  Jerung&X&X&X&X&X&X&-\\
  \hline
  Lekiu&X&X&X&X&not yet&-&-\\
  \hline  
  
 \end{tabular}
\caption{Features of the custom kernels (at least from what I remember)}
 \end{table}  
 \end{frame} 
 
 
 \begin{frame}{MultiROM capabilities}
 \begin{columns}
  \begin{column}{0.5\textwidth}
  \begin{figure}
  \includegraphics[scale=.08]{multirom}  
  \caption{List of ROMs}
  \end{figure}
  \end{column}
  
  \begin{column}{0.5\textwidth}
    \begin{figure}
   \includegraphics[scale=.08]{multirom2}    
   \caption{MultiROM settings}
    \end{figure}

    
  \end{column}
 \end{columns}
 \end{frame}

 %\section{Kernel configurations}
 
  \begin{frame}{Kernel configurations}
   \begin{figure}
   \includegraphics[scale=.1]{trickster}    
   \caption {\tiny{Using trickster to tune the kernel parameters, by sysfs interfacing}}
    \end{figure}
  \end{frame}

  
%  \subsection{I/O scheduler selector}
    \begin{frame}{I/O scheduler selector}
   \begin{figure}
   \includegraphics[scale=.1]{scheduler} 
   \caption {\tiny{Using trickster to select preferred I/O scheduler}}
   \end{figure}
     
    \end{frame}

%\subsection{CPU governors}    
\begin{frame}{CPU governors}
   \begin{figure}
   \includegraphics[scale=.1]{governor} 
   \caption {\tiny{Using trickster to select preferred CPU governors}}
   \end{figure}
     
    \end{frame}
    
    
 \begin{frame}{How to compile Android kernel?}
 \begin{itemize}
  \item Use desktop PC (Mine is i3, 16GB RAM, Ubuntu 14.04 LTS)
  \item Only works on 64-bit Linux
  \item Use cross compiler.. GCC for ARM
  \item Cross compiling in this case means compiling ARM kernel image on x86 (x64) based machine
  \begin{itemize}
   \item Default kernel sources somehow cannot use latest (bleeding edge) GCC (as for now 4.10). Need some patches
   to allow that.
  \end{itemize}
 \end{itemize}  
 \end{frame}
   
\begin{frame}{How to add features}
\begin{itemize}
 \item Use ``\texttt{patch}'' command by downloading intended patch from other source manually
 \item Use ``\texttt{git cherry pick}'' command 
\end{itemize}
\end{frame}

\begin{frame}[fragile]{Patch}
 \tiny \begin{Verbatim}
  wget -c https://github.com/engstk/l-preview/commit/8c203729fc0d4479b790408de2ac464745cc7769.patch
--2014-09-24 23:53:24--
https://github.com/engstk/l-preview/commit/8c203729fc0d4479b790408de2ac464745cc7769.patch
Resolving github.com (github.com)... 192.30.252.130
Connecting to github.com (github.com)|192.30.252.130|:443... connected.
HTTP request sent, awaiting response... 200 OK
Length: unspecified [text/plain]
Saving to: '8c203729fc0d4479b790408de2ac464745cc7769.patch'

    [  <=>                                                                                                               ] 20,146      59.6KB/s   in 0.3s   

2014-09-24 23:53:26 (59.6 KB/s) - '8c203729fc0d4479b790408de2ac464745cc7769.patch'saved [20146]

najmi@quds:~/lekiu-lprev-kernel$ patch -Np1 -i 8c203729fc0d4479b790408de2ac464745cc7769.patch 
patching file arch/arm/Kconfig
Hunk #1 succeeded at 2213 (offset 1 line).
patching file arch/arm/boot/compressed/head.S
patching file arch/arm/configs/hammerhead_defconfig
patching file arch/arm/include/asm/kexec.h
patching file arch/arm/kernel/machine_kexec.c
patching file arch/arm/kernel/relocate_kernel.S
patching file arch/arm/mach-msm/include/mach/memory.h
patching file arch/arm/mach-msm/lge/devices_lge.c
patching file arch/arm/mach-msm/restart.c
patching file include/linux/kexec.h
patching file kernel/kexec.c

 \end{Verbatim}

\end{frame}

\begin{frame}[fragile]{git cherry pick}
 \tiny \begin{Verbatim}        
najmi@quds:~/lekiu-lprev-kernel$ git fetch codeblue 
remote: Counting objects: 1073, done.
remote: Compressing objects: 100% (422/422), done.
remote: Total 1073 (delta 698), reused 974 (delta 649)
Receiving objects: 100% (1073/1073), 1.09 MiB | 443.00 KiB/s, done.
Resolving deltas: 100% (698/698), done.
From https://github.com/engstk/l-preview
 + 3043028...647125f code_blue-l -> codeblue/code_blue-l  (forced update)
 + 59063dc...647125f code_blue-l-beta -> codeblue/code_blue-l-beta  (forced update)

 najmi@quds:~/lekiu-lprev-kernel$ git cherry-pick 8c203729fc
[test cf6f160] Implement kexec-hardboot by. All work done by @Tasssadar
 Author: franciscofranco <franciscofranco.1990@gmail.com>
 11 files changed, 350 insertions(+), 7 deletions(-)
\end{Verbatim}
\end{frame}
   
\begin{frame}
 \begin{itemize}
  \item Since kernel sources are GPL, if you publish your release, you MUST publish your source codes
  \item People can view your codes/modifications or borrow/use them anyway they want, and they need to publish them as well  
 \end{itemize}
\end{frame}

\begin{frame}
%\frametitle{A first slide}

\begin{center}
\Huge Custom ROMs
\end{center}

\end{frame}


\begin{frame}{Custom ROM}
\begin{itemize}
 \item Custom ROMs are either AOSP or CM
 \item AOSP and CM ROM source are out there, many creative people do modifications on these sources
 \item Popular custom ROMs with relatively awesome team members and support many devices
 \begin{itemize}
  \item CyanogenMod
  \item Carbon ROM 
  \item Liquid Smooth
  \item Paranoid Android (PA)
  \item Mahdi ROM
  \item AOKP
 \end{itemize}
\end{itemize} 
\end{frame}


\begin{frame}
 \begin{itemize}
  \item Sources and modifications are updated in gerrits
  \item Cherry pick is also possible  
 \end{itemize}
\end{frame}

{\setbeamertemplate{logo}{}
\begin{frame}
\begin{columns}[t]
 
 \begin{column}{5cm}
  
 \begin{figure}
 \includegraphics[scale=0.1]{ls}
   \caption{\tiny{Liquid Smooth ROM, running Cempaka Kernel}}
 \end{figure}

 
\end{column}

 \begin{column}{5cm}
 \begin{figure}
 \includegraphics[scale=0.1]{carbon}
   \caption{\tiny{Carbon ROM, running Blue Kelisa Kernel}}
 \end{figure}
 \end{column}

\end{columns}
 \end{frame}
}

\begin{frame}{Source sync}
\begin{itemize}
 \item ROMs are using ``repo'' command to sync all changes from the upstream
 \item Updates are based from the repositories defined in the manifest file
\end{itemize} 
\end{frame}

\begin{frame}[fragile]{Repo sync}
\tiny \begin{verbatim}
najmi@quds:~/rom/aicp$ repo sync
remote: Counting objects: 3, done.
remote: Compressing objects: 100% (3/3), done.
remote: Total 3 (delta 0), reused 3 (delta 0)
Unpacking objects: 100% (3/3), done.
From https://github.com/AICP/platform_manifest
   9ee1fc3..d607bb2  kitkat     -> origin/kitkat
project .repo/manifests/
Updating 9ee1fc3..d607bb2
Fast-forward
 default.xml | 2 +-
 1 file changed, 1 insertion(+), 1 deletion(-)

Fetching project platform/packages/apps/Provision
Fetching project CyanogenMod/android_hardware_ti_wpan
Fetching project omnirom/android_packages_wallpapers_PhaseBeam
Fetching project platform/external/tinyxml
...............
remote: Counting objects: 12, done
remote: Finding sources: 100% (12/12)
remote: Total 12 (delta 0), reused 12 (delta 0)
remote: Counting objects: 12, done
Unpacking objects: 100% (12/12), done. 
From https://android.googlesource.com/platform/external/eigen
\end{verbatim}
\end{frame}

\begin{frame}
Example of extra features in custom ROMS
 \begin{itemize}
  \item Hover
  \item PIE
  \item Halo
  \item Customization of date display
 \end{itemize}

\end{frame}

   
   
   {\setbeamertemplate{logo}{}
\begin{frame}
\begin{columns}[t]
 
 \begin{column}{5cm}
  
 \begin{figure}
 \includegraphics[scale=0.1]{pie}
   \caption{\tiny{PIE}}
 \end{figure}

 
\end{column}

 \begin{column}{5cm}
 \begin{figure}
 \includegraphics[scale=0.1]{hover}
   \caption{\tiny{Hover}}
 \end{figure}
 \end{column}

\end{columns}
 \end{frame}
}


\begin{frame}{END}
%\frametitle{A first slide}

\begin{center}
\Large \texttt{najmi.zabidi@gmail.com}
\end{center}

\end{frame}
\end{document}



